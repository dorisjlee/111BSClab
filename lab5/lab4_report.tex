\documentclass[authoryear, 12pt,5p, times]{elsarticle}
%\usepackage[hypcap]{caption}
%\geometry{margin=0.95in,top=1.4in,bottom=1.4in}
\geometry{margin=1.1in,top=1.5in,bottom=1.5in}
\usepackage{float}
\usepackage{amsmath}
\usepackage[hidelinks]{hyperref} 
 \usepackage{gensymb}
\usepackage{subcaption}
\usepackage{url}
%\renewcommand\thefootnote{\fnsymbol{\dagger}}
\usepackage[symbol*]{footmisc}
\makeatletter
\newcommand{\rpm}{\raisebox{.3ex}{$\scriptstyle\pm$}}
\begin{document}
\begin{frontmatter}
\title{JFET Circuits II}
\author{\today \quad \\Jung Lin (Doris) Lee [Lab Partner: Leah Tom]\\Prof. William Holzapfel, GSI Thomas Darlington, Thomas Mittiga, John Groh,  \\Victoria Xu, Jonathan Ma, Francisco Monsalve, Xiaofei Zhou\vspace{-30pt}}	 
\end{frontmatter}
\section*{Introduction\label{intro}}
 
\section*{5.1}
\section*{5.2}
 
\section*{5.3}
 
 \section*{Conclusion}
In this lab, we investigated the characteristics of  
%Diodes are useful --- their one directional characteristic
\section*{Acknowledgments}
\begin{footnotesize}
The author would like to acknowledge support from the GSI in this lab in addressing our questions about the lab and with the handling of liquid nitrogen. I would also like to thank my partner, Leah Tom, for helpful discussion and collaboration that helped this work. We also appreciate Sissi Wang for providing us with guidance on question 3.5, 3.8, and 3.9.
\end{footnotesize}
  \section*{References}
 \begin{footnotesize}
 \begin{itemize}
 \item Horowitz, Paul, and Winfield Hill. \textit{The Art of Electronics}. Cambridge: Cambridge UP, 1989. Print.
 \item ``Lab 5 - JFET Circuits II. " \textit{Donald A. Glaser Advanced Lab.} Regents of the University of California, n.d. Web. 01 Feb. 2015.
 \end{itemize} 
  \end{footnotesize}

\end{document}
