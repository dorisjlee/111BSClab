%Full Report = Includes Introduction and Conclusion. Re-write experiments, integrate graphs, analysis, and comments to
%make a clean easy to read report. Your report must include carbonless copies of in-lab notes. Type written reports.
\documentclass[10pt,letterpaper,oneside] {article}
\usepackage{graphicx}
\usepackage{amsmath}
\usepackage[normalem]{ulem}
\usepackage[hidelinks]{hyperref}
\usepackage{natbib}
\hypersetup{colorlinks,urlcolor=blue}
%\usepackage[labelformat=empty]{caption}
% hack into hyperref
\makeatletter
\usepackage[margin=1.0in]{geometry}
\newcommand{\tab}[1]{\hspace{.26\textwidth}\rlap{#1}}
%\renewcommand{\labelitemii}{$\cdot$}
\renewcommand{\labelitemiii}{$\diamond$}
\begin{document}
\title{Introductory Experiments and Linear Circuits I}
\author{\quad \\Jung Lin (Doris) Lee [Lab Partner: Leah Tom]\\Prof. William Holzapfel, GSI Thomas Darlington, Thomas Mittiga, John Groh,  \\Victoria Xu, Jonathan Ma, Francisco Monsalve, Xiaofei Zhou}
\maketitle
	\begin{abstract}
	In this lab, we explore ----- BSC 
	\end{abstract}


\section{Introduction}
 
%\begin{figure}[h!]
%\includegraphics[width=400pt]{figure/decisiontree}
%\label{decisiontree}
%\end{figure}

\section{Keithley 2110 Digital Multimeter3}
- uncertainty
The range should be adjusted  suitable range within --- for each measurement , within order of magnitude. \footnote{Too large a range will result in the error ``OVLRD" (overload) and too low will cause ---}
\section{BSC Laboratory Breadboard Box}

\section{Digital Oscilloscope}
voltage on the vertical axes and time on the horizontal axes
\subsection{Tune-able Parameters and useful functions}
\begin{itemize}
\item AC/DC Setting : (See Sec.\ref{sec:acdc})
\item Scale: Vertical and horizontal zoom in ; adjust accordingly to --- window that best captures
\item Measurement: useful quantities 
\end{itemize}
\section{Arbitrary Waveform Function Generator}
\section{Frequency and time measurements}
\subsection{AC measurement}\label{sec:acdc}
\section{Thevinin Equivalence}
 \section{Conclusion}
\section{Acknowledgments}
\bibliography{references}
\bibliographystyle{plain}
\end{document}